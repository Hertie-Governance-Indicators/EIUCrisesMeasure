\documentclass[]{article}
\usepackage{lmodern}
\usepackage{setspace}
\setstretch{1.5}
\usepackage{amssymb,amsmath}
\usepackage{ifxetex,ifluatex}
\usepackage{fixltx2e} % provides \textsubscript
\ifnum 0\ifxetex 1\fi\ifluatex 1\fi=0 % if pdftex
  \usepackage[T1]{fontenc}
  \usepackage[utf8]{inputenc}
\else % if luatex or xelatex
  \ifxetex
    \usepackage{mathspec}
    \usepackage{xltxtra,xunicode}
  \else
    \usepackage{fontspec}
  \fi
  \defaultfontfeatures{Mapping=tex-text,Scale=MatchLowercase}
  \newcommand{\euro}{€}
\fi
% use upquote if available, for straight quotes in verbatim environments
\IfFileExists{upquote.sty}{\usepackage{upquote}}{}
% use microtype if available
\IfFileExists{microtype.sty}{%
\usepackage{microtype}
\UseMicrotypeSet[protrusion]{basicmath} % disable protrusion for tt fonts
}{}
\usepackage[margin=1in]{geometry}
\ifxetex
  \usepackage[setpagesize=false, % page size defined by xetex
              unicode=false, % unicode breaks when used with xetex
              xetex]{hyperref}
\else
  \usepackage[unicode=true]{hyperref}
\fi
\hypersetup{breaklinks=true,
            bookmarks=true,
            pdfauthor={Christopher Gandrud, Sahil Deo, Christian Franz, and Mark Hallerberg},
            pdftitle={Measuring Real-time Perceptions of Financial Market Stress},
            colorlinks=true,
            citecolor=blue,
            urlcolor=blue,
            linkcolor=magenta,
            pdfborder={0 0 0}}
\urlstyle{same}  % don't use monospace font for urls
\setlength{\parindent}{0pt}
\setlength{\parskip}{6pt plus 2pt minus 1pt}
\setlength{\emergencystretch}{3em}  % prevent overfull lines
\setcounter{secnumdepth}{0}

%%% Use protect on footnotes to avoid problems with footnotes in titles
\let\rmarkdownfootnote\footnote%
\def\footnote{\protect\rmarkdownfootnote}

%%% Change title format to be more compact
\usepackage{titling}

% Create subtitle command for use in maketitle
\newcommand{\subtitle}[1]{
  \posttitle{
    \begin{center}\large#1\end{center}
    }
}

\setlength{\droptitle}{-2em}
  \title{Measuring Real-time Perceptions of Financial Market Stress}
  \pretitle{\vspace{\droptitle}\centering\huge}
  \posttitle{\par}
  \author{Christopher Gandrud, Sahil Deo, Christian Franz, and Mark Hallerberg}
  \preauthor{\centering\large\emph}
  \postauthor{\par}
  \date{}
  \predate{}\postdate{}



\begin{document}

\maketitle


\textbf{Note: This work is in the early stages of development. It will
be updated significantly.}

\textbf{Abstract}\footnote{Please contact Christopher Gandrud
  (\href{mailto:gandrud@hertie-school.org}{\nolinkurl{gandrud@hertie-school.org}}).
  Thank you to Ronen Palan for helpful comments. All data and
  replication material can be found at:
  \url{https://github.com/christophergandrud/EIUCrisesMeasure}.}

\begin{quote}
This paper describes the motivation and construction of a new measure of
real-time perceptions of financial market stress based on machine
classification of \href{http://www.eiu.com/}{Economist Intelligence
Unit} monthly country reports.
\end{quote}

Why and how do politicians respond to financial market stress? This
question has attracted considerable attention recently following the
2007-2009 financial crisis and earlier following the late-1990s Asian
financial crisis. However, virtually all of this research lacks a
crucial variable: a real-time indication of the level of financial
market stress that policy-makers believed that they faced. To understand
why politicians made a given policy choice, we need to have a measure of
what problems they faced.

Most research has used \emph{post-hoc} assessments of banking crisis as
a second-best alternative. However, this presents clear problems.
Chiefly, using such measures creates clear selection bias as stress that
politicians responded to effectively will not be selected. In addition,
these measures are typically binary and so give now indication of stress
intensity. The measures are also at gross intervals, typically yearly,
prohibiting sub-annual analysis.

In this paper we aim to overcome these problems by develop a new index
of real-time perceptions of financial market stress. The index is
created using a kernel principal component analysis of monthly Economist
Intelligence Unit (EIU) reports. This measure should supplant previous
second-best measures of financial market stress by researchers aiming to
understand why and how policy-makers respond to financial crisis.

We start the paper by detailing our motivation for creating a real-time
index of perceptions of financial market stress. We then discuss the
construction of the Index and compare it to widely used previous
measures of financial market stress. {[}WOULD BE NICE TO HAVE A
REPLICATION OF AN IMPORTANT PAPER{]}.

\subsection{Motivation}\label{motivation}

Researchers have tended to rely on two data sources for cross-country
information on when a country is facing a financial crisis: Laeven and
Valencia (2013) and Reinhart and Rogoff (2009). Knowing when crises
started (and when they have ended) is crucial for research trying to
understand issues such as how crises affect economic output, how
governments choose to respond to financial market distress, and what the
fiscal costs of financial crises are.

There are a number of problems with these indicators. Unlike economic
recessions, financial crises are poorly defined in previous sources.
This contributes to large inconsistencies between the timing of crises
in the Laeven and Valencia (2013) and Reinhart and Rogoff (2009) data
sets (Chaudron and Haan 2014). For example, Japan is labeled as having a
crisis between 1997 and 2001 by the former, but 1992-1997 in the latter.
Gandrud and Hallerberg (2015) also find that there are significant
difference in crisis timing between different versions of the Laeven and
Valencia (2013) data. Crises are also identified by researchers who know
what happened. Financial market stress that is addressed well by
policymakers, preventing a major crisis, may therefore not be included.
Similarly, stress that is temporarily dampened through unsustainable
policy measures, only to flare up later, is not clearly recorded. This
makes it difficult to adequately study why and how politicians respond
to financial market stress. Related to this, current measures are
dichotomous thus errors have large consequences for creating bias when
used in econometric models. They also do not give any indication of how
severe a crisis is.

Overall, we lack the continuous real-time measure of financial market
stress that we need to be able to adequately examine why and how
policy-makers respond to financial market problems.

There have been a number of recent attempts to create crisis measures
that overcome these issues. Building on {Von Hagen} and Ho (2007), Jing
et al. (2015) developed am index of money market pressure based on
changes in short-term interest rates and stocks of central bank
reserves. However, this measure conflates distress and policy responses,
assuming central banks use the same reaction function to increased
demand for liquidity. Rosas (2009) developed a dynamic latent trait
model of banking system distress. However, his measure relies on
nationally reported data to the IMF's International Financial
Statistics, which Copelovitch, Gandrud, and Hallerberg (2015) show can
be endogenous to financial market distress.

C. Romer and Romer (2014) aimed to address this issue by manually
classifying 24 countries on a 15 point scale capturing the cost of
credit intermediation. They code countries using information from OECD
semi-annual \emph{Economic Outlook} reports from 1967 to 2007. Relying
on contemporaneous reports allows for the construction of a real-time
measure of credit market distress. This would allow us to examine policy
choices that head off trouble or unsustainably prolong brewing
difficulties. Their, relatively, continuous measure gives an indication
of market distress intensity.

Their approach could be improved in a number of key ways. First, they
are necessarily limited to the relatively small sample of OECD
countries. Second, their measure is laborious to create and update.
Third, the scale is created by simply summing

{[}COMPLETE{]}

\subsection{~Creating the Perceptions of Financial Market Stress
Index}\label{creating-the-perceptions-of-financial-market-stress-index}

We propose a new method of estimating a real-time measure of perceptions
of financial market stress. The index is created with kernel principle
component analysis (Scholkopf, Smola, and Muller 1998; Spirling 2012) of
monthly country reports from the \textbf{Economist Intelligence
Unit}.\footnote{See \url{http://www.eiu.com/}. Accessed May 2015.}

\subsubsection{Why the EIU?}\label{why-the-eiu}

The EIU is the product of a an analysis of real-time, third-party
assessments of financial market conditions reported monthly or quarterly
(depending on the country). These reports are both a summary of
real-time information and forecasts on counties' economic conditions as
well as a channel through which this information is disseminated to
public and private actors. Together, the reports create a very large
corpus (more than 20,000 texts from 1997 through 2011 {[}CHECK{]}) of
monthly reports for more than 100 countries. As the texts generally
follow the same format and style, they contain directly comparable
assessments of economic conditions monthly across the globe for a
significant time span. In contrast, the OECD \textbf{Economic Outlook}
provides comparable reports for a very small number of wealthy countries
on a bi-annual basis.

\subsubsection{Summarising Financial Market Stress in the
EIU}\label{summarising-financial-market-stress-in-the-eiu}

Our aim is to create an index that classifies financial conditions on a
continuous more-stressed/less stressed spectrum. So we clearly need an
efficient way to summarize the vast quantity of information in the EIU
reports. To do this we first collected and processed the texts. Then we
used kernel principal component analysis to summarize the texts into a
dimension of financial market stress. We rescaled the Index to ease
interpretation. Finally, we used a number of strategies to examine the
Index's validity.

\paragraph{Text selection}\label{text-selection}

EIU reports contain assessments of a wide range of countries' economies,
not just their financial system. So, our first step was to select the
portions of the EIU texts that contained relevant information about
countries' banking and financial systems. We collected the parsed
reports--the reports were in HTML format. We then extracted the portions
of the texts--headlines and paragraphs--that contained at least one of a
number of keywords concerning banking and financial markets.\footnote{The
  keywords included: \emph{bail-out}, \emph{bailout}, \emph{balance
  sheet}, \emph{bank}, \emph{credit}, \emph{crunch}, \emph{default},
  \emph{financial}, \emph{lend}, \emph{loan}, \emph{squeeze} {[}MAKE
  SURE TO UPDATE{]}. These keywords were adapted from those used by C.
  Romer and Romer (2014) and are intended to select passages that
  discusses credit market conditions.} Due to a significant change in
how the reports were constructed in 2003, we also selected only texts
from 2003 in order to maintain comparability across the time-series.

We then preprocessed these texts using standard techniques (see Grimmer
and Stewart 2013).\footnote{All preprocessing was done using the
  \texttt{tm} package (Feinerer and Hornik 2015) in R (R Core Team
  2015).} This involved removing common English words, such as `was' and
`its', stemming the words so that different variants of the same word
are grouped together {[}MAYBE DON'T DO{]}, removing extra whitespace
between the words, converting the words to lower case {[}MAYBE DON'T
DO{]}, removing punctuation and numbers. Finally, we dropped texts that
included very few words (less than six). Including these texts prevented
the estimation of the kernel PCA model.

\paragraph{Kernel Principal Component
Analysis}\label{kernel-principal-component-analysis}

\paragraph{Characteristics of the
Index}\label{characteristics-of-the-index}

\subparagraph{Developed vs.~Developing
countries}\label{developed-vs.developing-countries}

An important finding from examining the Index is that there is a clear
difference in the level of perceived financial market stress in
developed and developing countries. Notably, developing countries often
have scores well above 0.5, while many developed countries only reach
this level during financial crises. Developing countries often lack
strong financial institutions and systems {[}CITE{]}, so we should
expect them to face generally tighter credit market conditions than
developed countries. Formal financial markets are less important for
developing countries' economies {[}CITE{]}.

These observations should lead to an important refinement to how the
Index should be interpreted and how it should be used in empirical work.
First, the Index measures banking market conditions, but not crisis
directly. Instead, perceived crisis is likely the result of an
interaction between the Index and the importance of financial markets
for sustaining a country's economy. Though policy-makers in developing
economies face generally tight credit market conditions, these
persistent conditions likely do not threaten the wider \emph{status quo}
economy. As such, we would not expect significant policy responses to
address financial market stress in these places. Conversely, tightening
of credit market conditions in a developed, financialised economy would
likely have large negative implications for the wider economy. So, we
would expect these politicians to respond to worsening credit market
conditions.

Previous measures of financial market distress and crises have generally
been unable to explore this possible interaction. \emph{Post-hoc}
measures of crisis in particular capture the outcome of this process,
rather than the process itself.

\subsection{~Comparison to Other Crisis
Measures}\label{comparison-to-other-crisis-measures}

\subsection{Conclusions and Possible Future
Work}\label{conclusions-and-possible-future-work}

\subsection*{References}\label{references}
\addcontentsline{toc}{subsection}{References}

Chaudron, Raymond, and Jakob de Haan. 2014. ``Dating Banking Crises
Using Incidence and Size of Bank Failures: Four Crises Reconsidered.''
\emph{Journal of Financial Stability}, 1--34.

Copelovitch, Mark, Christopher Gandrud, and Mark Hallerberg. 2015.
``Financial Regulatory Transparency, International Institutions, and
Borrowing Costs.'' \emph{Working Paper}.

Feinerer, Ingo, and Kurt Hornik. 2015. \emph{tm: Text Mining Package}.
\url{http://CRAN.R-project.org/package=tm}.

Gandrud, Christopher, and Mark Hallerberg. 2015. ``When All Is Said and
Done: Updating 'Elections, Special Interests, and Financial Crisis'.''
\emph{Research and Politics}.

Grimmer, Justin, and Brandon M Stewart. 2013. ``Text as Data: The
Promise and Pitfalls of Automatic Content Analysis Methods for Political
Texts.'' \emph{Political Analysis} 21 (3): 267--97.

Jing, Zhongbo, Jakob de Haan, Jan Jacobs, and Haizhen Yang. 2015.
``Identifying Banking Crises Using Money Market Pressure: New Evidence
for a Large Set of Countries.'' \emph{Journal of Macroeconomics} 43 (C):
1--20.

Laeven, Luc, and Fabi{á}n Valencia. 2013. ``Systemic Banking Crisis
Database.'' \emph{IMF Economic Review} 61 (2): 225--70.

R Core Team. 2015. \emph{R: A Language and Environment for Statistical
Computing}. Vienna, Austria: R Foundation for Statistical Computing.
\url{http://www.R-project.org/}.

Reinhart, Carmen, and Kenneth Rogoff. 2009. \emph{This Time Is
Different: Eight Centuries of Financial Folly}. Princeton: Princeton
University Press.

Romer, Christina, and David Romer. 2014. ``New Evidence on the Impact of
Financial Crises in Advanced Countries,'' Nov, 1--65.

Rosas, Guillermo. 2009. ``Dynamic Latent Trait Models: An Application to
Latin American Banking Crises.'' \emph{Electoral Studies} 28: 375--87.

Scholkopf, B., A. Smola, and K. Muller. 1998. ``Nonlinear Component
Analysis as a Kernel Eigenvalue Problem.'' \emph{Neural Computation} 10:
1299--1319.

Spirling, Arthur. 2012. ``U.S. Treaty Making with American Indians:
Institutional Change and Relative Power, 1784-1911.'' \emph{American
Journal of Political Science} 56 (1): 84--97.

{Von Hagen}, Jorgen, and T. Ho. 2007. ``Money Market Pressure and the
Determinants of Banking Crises.'' \emph{Journal of Money, Credit, and
Banking} 39 (5): 1037--66.

\end{document}
