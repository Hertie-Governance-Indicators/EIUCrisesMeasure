\documentclass[]{article}
\usepackage{lmodern}
\usepackage{amssymb,amsmath}
\usepackage{ifxetex,ifluatex}
\usepackage{fixltx2e} % provides \textsubscript
\ifnum 0\ifxetex 1\fi\ifluatex 1\fi=0 % if pdftex
  \usepackage[T1]{fontenc}
  \usepackage[utf8]{inputenc}
\else % if luatex or xelatex
  \ifxetex
    \usepackage{mathspec}
    \usepackage{xltxtra,xunicode}
  \else
    \usepackage{fontspec}
  \fi
  \defaultfontfeatures{Mapping=tex-text,Scale=MatchLowercase}
  \newcommand{\euro}{€}
\fi
% use upquote if available, for straight quotes in verbatim environments
\IfFileExists{upquote.sty}{\usepackage{upquote}}{}
% use microtype if available
\IfFileExists{microtype.sty}{%
\usepackage{microtype}
\UseMicrotypeSet[protrusion]{basicmath} % disable protrusion for tt fonts
}{}
\usepackage[margin=1in]{geometry}
\ifxetex
  \usepackage[setpagesize=false, % page size defined by xetex
              unicode=false, % unicode breaks when used with xetex
              xetex]{hyperref}
\else
  \usepackage[unicode=true]{hyperref}
\fi
\hypersetup{breaklinks=true,
            bookmarks=true,
            pdfauthor={},
            pdftitle={Machine Index of Financial Market Stress (MIFMS)},
            colorlinks=true,
            citecolor=blue,
            urlcolor=blue,
            linkcolor=magenta,
            pdfborder={0 0 0}}
\urlstyle{same}  % don't use monospace font for urls
\setlength{\parindent}{0pt}
\setlength{\parskip}{6pt plus 2pt minus 1pt}
\setlength{\emergencystretch}{3em}  % prevent overfull lines
\setcounter{secnumdepth}{0}

%%% Use protect on footnotes to avoid problems with footnotes in titles
\let\rmarkdownfootnote\footnote%
\def\footnote{\protect\rmarkdownfootnote}

%%% Change title format to be more compact
\usepackage{titling}

% Create subtitle command for use in maketitle
\newcommand{\subtitle}[1]{
  \posttitle{
    \begin{center}\large#1\end{center}
    }
}

\setlength{\droptitle}{-2em}
  \title{Machine Index of Financial Market Stress (MIFMS)}
  \pretitle{\vspace{\droptitle}\centering\huge}
  \posttitle{\par}
  \author{}
  \preauthor{}\postauthor{}
  \date{}
  \predate{}\postdate{}



\begin{document}

\maketitle


\textbf{Note: This work is in the early stages of development. It will
be updated significantly.}

\begin{quote}
This paper describes the motivation and construction of a new measure of
financial market stress based on machine classification of
\href{http://www.eiu.com/}{Economist Intelligence Unit} monthly country
reports.
\end{quote}

\subsection{Motivation}\label{motivation}

Researchers have tended to rely on two data sources for cross-country
information on when a country is facing a financial crisis: Laeven and
Valencia (2013) and Reinhart and Rogoff (2009). Knowing when crises
started (and when they have ended) is crucial for research trying to
understand issues such as how crises affect economic output, how
governments choose to respond to financial market distress, and what the
fiscal costs of financial crises are.

There are a number of problems with these indicators. Unlike economic
recessions, financial crises are poorly defined in previous sources.
This contributes to large inconsistencies between the timing of crises
in the Laeven and Valencia (2013) and Reinhart and Rogoff (2009) data
sets (Chaudron and Haan 2014). For example, Japan is labeled as having a
crisis between 1997 and 2001 by the former, but 1992-1997 in the latter.
Gandrud and Hallerberg (forthcoming) also find that there are
significant difference in crisis timing between different versions of
the Laeven and Valencia (2013) data. Crises are also identified by
researchers who know what happened. Financial market stress that is
addressed well by policymakers, preventing a major crisis, may therefore
not be included. Similarly, stress that is temporarily dampened through
unsustainable policy measures, only to flare up later, is not clearly
recorded. This makes it difficult to adequately study why and how
politicians respond to financial market stress. Related to this, current
measures are dichotomous thus errors have large consequences for
creating bias when used in econometric models. They also do not give any
indication of how severe a crisis is.

There have been a number of recent attempts to create crisis measures
that overcome these issues. Building on {Von Hagen} and Ho (2007), Jing
et al. (2015) developed am index of money market pressure based on
changes in short-term interest rates and stocks of central bank
reserves. However, this measure conflates distress and policy responses,
assuming central banks use the same reaction function to increased
demand for liquidity. Rosas (2009) developed a dynamic latent trait
model of banking system distress. However, his measure relies on
nationally reported data to the IMF's International Financial
Statistics, which Copelovitch, Gandrud, and Hallerberg (2015) show can
be endogenous to financial market distress.

C. Romer and Romer (2014) aimed to address this issue by manually
classifying 24 countries on a 15 point scale capturing the cost of
credit intermediation. They code countries using information from OECD
semi-annual \emph{Economic Outlook} reports from 1967 to 2007. Relying
on contemporaneous reports allows for the construction of a real-time
measure of credit market distress. This would allow us to examine policy
choices that head off trouble or unsustainably prolong brewing
difficulties. Their, relatively, continuous measure gives an indication
of market distress intensity.

Their approach could be improved in a number of key ways. First, they
are necessarily limited to the relatively small sample of OECD
countries. Second, their measure is laborious to create and update.
Third, the scale is created by simply summing

{[}COMPLETE{]}

\subsection{~Machine Index of Financial Market
Stress}\label{machine-index-of-financial-market-stress}

We propose a new method of estimating financial market stress that
addresses many of the problems in previous indexes. The index is created
with kernel principle component analysis (Scholkopf, Smola, and Muller
1998; Spirling 2011) of monthly country reports from the Economist
Intelligence Unit. The index is the product of a an analysis of
real-time, third-party assessments of financial market conditions
reported monthly.

Monthly

\subsection*{References}\label{references}
\addcontentsline{toc}{subsection}{References}

Chaudron, Raymond, and Jakob de Haan. 2014. ``Dating Banking Crises
Using Incidence and Size of Bank Failures: Four Crises Reconsidered.''
\emph{Journal of Financial Stability}, 1--34.

Copelovitch, Mark, Christopher Gandrud, and Mark Hallerberg. 2015.
``Financial Regulatory Transparency, International Institutions, and
Borrowing Costs.'' \emph{Working Paper}.

Gandrud, Christopher, and Mark Hallerberg. forthcoming. ``When All Is
Said and Done: Updating 'Elections, Special Interests, and Financial
Crisis'.'' \emph{Research and Politics}.

Jing, Zhongbo, Jakob de Haan, Jan Jacobs, and Haizhen Yang. 2015.
``Identifying Banking Crises Using Money Market Pressure: New Evidence
for a Large Set of Countries.'' \emph{Journal of Macroeconomics} 43 (C):
1--20.

Laeven, Luc, and Fabi{á}n Valencia. 2013. ``Systemic Banking Crisis
Database.'' \emph{IMF Economic Review} 61 (2): 225--70.

Reinhart, Carmen, and Kenneth Rogoff. 2009. \emph{This Time Is
Different: Eight Centuries of Financial Folly}. Princeton: Princeton
University Press.

Romer, {Christina D.}, and {David H.} Romer. 2014. ``New Evidence on the
Impact of Financial Crises in Advanced Countries,'' Nov, 1--65.

Rosas, Guillermo. 2009. ``Dynamic Latent Trait Models: An Application to
Latin American Banking Crises.'' \emph{Electoral Studies} 28: 375--87.

Scholkopf, B., A. Smola, and K. Muller. 1998. ``Nonlinear Component
Analysis as a Kernel Eigenvalue Problem.'' \emph{Neural Computation} 10:
1299--1319.

Spirling, Arthur. 2011. ``U.S. Treaty Making with American Indians:
Institutional Change and Relative Power, 1784-1911.'' \emph{American
Journal of Political Science} 56 (1): 84--97.

{Von Hagen}, Jorgen, and T. Ho. 2007. ``Money Market Pressure and the
Determinants of Banking Crises.'' \emph{Journal of Money, Credit, and
Banking} 39 (5): 1037--66.

\end{document}
