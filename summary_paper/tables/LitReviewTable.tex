\begin{tabular}{ m{2.5cm} m{1.75cm} m{6.25cm} m{2.5cm}}
    \hline
    Work & Crisis Type & Key Arguments/Findings & Crisis Data Sources \\
    \hline\hline

    %%% Broz
    \cite{broz2013} & Banking crisis & - In OECD countries right-wing governments pursue policies that lead to financial instability. Voters respond to resulting crises by voting in left-wing governments. & \cite{Reinhart2009,Laeven2012} \\[0.25cm]\hline

    %%% Galasso
    \cite{galasso2014} & Financial and economic crises & Governments respond to financial crises by increasing regulation. & Dummy based on OECD output gap below -3.4\% \\[0.25cm]\hline

    %%% Gandrud
    \cite{Gandrud2013,Gandrud2014} & Banking crises & - Best practice financial governance institutional designs are more likely to be adopted during crises when there is high uncertainty about policy choices and outcomes. & \cite{Laeven2008,ReinhartRog2010} \\[0.25cm]\hline

    %%% Hallerberg Scart
    \cite{HallerbergScartForthcoming} & Banking, debt crises & - Banking crises reduce the probability of fiscal reforms, but the longer a crisis lasts and if it becomes a sovereign debt crisis the the probability of reform increases.

    - Countries with more personalistic voting are more likely to reform. & \cite{Laeven2012} for Latin American countries \\[0.25cm]\hline

    %%% Hallerberg Wehner
    \cite{Hallerberg2013} & Banking, currency, debt crises & - Some evidence that more technically competent ministers of finance are appointed during debt crises. Not much robust evidence for other effects of crisis on the technical competency of economic policy-makers. & \cite{Laeven2012}  \\[0.25cm]\hline

    %%% Hicken et al.
    \cite{Hicken2005} (2005) & Growth shocks & - The size of the winning coalition is positively associated with growth recoveries following forced devaluations. & Own data aggregated from multiple sources \\[0.25cm]\hline

    %%%% Keefer
    \cite{Keefer2007} & Banking crises & - Higher electoral competitiveness leads to faster and less costly crisis responses.

    - Checks and balances not associated with crisis policy choices or outcomes. & Modified \cite{Honohan2003} \\[0.25cm]\hline

    %%% Kleibl
    \cite{Kleibl2013} & Banking crisis & - Responses to regulatory failures are conditioned by the level of public ownership in the banking sector. & \cite{Laeven2010,Reinhart2009} for OECD countries \\[0.25cm]\hline

    %%% MacIntyre
    \cite{MacIntyre2001} & Financial crises & - U-shaped relationship between veto players and crisis outcomes & Own data aggregated from multiple sources \\[0.25cm]\hline

    %%% Rodrick
    \cite{Rodrick1999} & Growth shock & - Many veto players, if organized to manage conflicts, will result in more appropriate and quickly implemented crisis management policies. & Own data aggregated from multiple sources \\[0.25cm]\hline

    %%%% Rosas
    \cite{Rosas2006,Rosas2009} & Banking crisis & - Democratic regimes have fewer bailouts.

    - Central bank independence and transparency lead to fewer bailouts. & Modified \cite{Honohan2000} \\[0.25cm]\hline

    %%% Satyanath
    \cite{Satayanath2006} & Banking crises & - Executives without `banking cronies' and that are not prevented from appointing their own bureaucrats by many veto players are more likely to have stringent financial regulation that prevents crises. & Case studies of 7 East Asian countries \\[0.25cm]\hline

    %%% Wibbels and Roberts
    \cite{Wibbels2010} & Currency, growth, \& fiscal crises & - Unions and strong left parties are more associated with crises, though combined strong unions-left parties may alleviate inflationary crises. & Own data aggregated from multiple sources for 17 Latin American countries \\[0.25cm]\hline


    \hline
\end{tabular}
