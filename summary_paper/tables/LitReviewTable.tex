\begin{tabular}{ m{2.5cm} m{1.75cm} m{6.25cm} m{2.5cm} m{1.75cm}}
    \hline
    Work & Crisis Type & Key Arguments/Findings & Crisis Data Sources & Observation Period \\
    \hline\hline

    %%% Gandrud
    Gandrud (2013, 2014) & Banking crises & - Best practice financial governance institutional designs are more likely to be adopted during crises when there is high uncertainty about policy choices and outcomes. & Laeven \& Valencia (2008), Reinhart \& Rogoff (2010) & Late 1980s-2007 \\[0.25cm]\hline

    %%% Hallerberg Scart
    Hallerberg \& Scartascini (2013) & Banking, debt crises & - Banking crises reduce the probability of fiscal reforms, but the longer a crisis lasts and if it becomes a sovereign debt crisis the the probability of reform increases.

    - Countries with more personalistic voting are more likely to reform. & Laeven \& Valencia (2012) for Latin American countries & 1975-2005 \\[0.25cm]\hline

    %%% Hallerberg Wehner
    Hallerberg \& Wehner (2013) & Banking, currency, debt crises & - Some evidence that more technically competent ministers of finance are appointed during debt crises. Not much robust evidence for other effects of crisis on the technical competency of economic policy-makers. & Laeven \& Valencia (2012) & 1975-2010 \\[0.25cm]\hline

    %%% Hicken et al.
    Hicken, Satyanath, \& Sergenti (2005) & Growth shocks & - The size of the winning coalition is positively associated with growth recoveries following forced devaluations. & Own data aggregated from multiple sources & 1990s-2002 \\[0.25cm]\hline

    %%%% Keefer
    Keefer (2007) & Banking crises & - Higher electoral competitiveness leads to faster and less costly crisis responses.

    - Checks and balances not associated with crisis policy choices or outcomes. & Modified Honohan \& Klingebiel (2003) & 1975-2000 \\[0.25cm]\hline

    %%% Kleibl
    Kleibl (2013) & Banking crisis & - Responses to regulatory failures are conditioned by the level of public ownership in the banking sector. & Laeven \& Valencia (2010), Reinhart \& Rogoff (2009) for OECD countries & 1975-2010 \\[0.25cm]\hline

    %%% MacIntyre
    MacIntyre (2001) & Financial crises & - U-shaped relationship between veto players and crisis outcomes & Own data aggregated from multiple sources & 1997-1998 \\[0.25cm]\hline

    %%% Rodrick
    Rodrick (1991) & Growth shock & - Many veto players, if organized to manage conflicts, will result in more appropriate and quickly implemented crisis management policies. & Own data aggregated from multiple sources & 1960-1975 \& 1975-1989 \\[0.25cm]\hline

    %%%% Rosas
    Rosas (2006, 2009) & Banking crisis & - Democratic regimes have fewer bailouts.

    - Central bank independence and transparency lead to fewer bailouts. & Modified Honohan \& Klingebiel (2000) & 1980-1998 \\[0.25cm]\hline

    %%% Satyanath
    Satyanath (2006) & Banking crises & - Executives without `banking cronies' and that are not prevented from appointing their own bureaucrats by many veto players are more likely to have stringent financial regulation that prevents crises. & Case studies of 7 East Asian countries & 1997 Asian Financial Crisis \\[0.25cm]\hline

    %%% Wibbels and Roberts
    Wibbels \& Roberts (2010) & Currency, growth, \& fiscal crises & - Unions and strong left parties are more associated with crises, though combined strong unions-left parties may alleviate inflationary crises. & Own data aggregated from multiple sources for 17 Latin American countries & 1980-2006 \\[0.25cm]\hline


    \hline
\end{tabular}
